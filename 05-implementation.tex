The prototype implementation of Noria with partial state consists of 65k lines
of Rust. It can operate on a single server or across a cluster of servers.

The source code is available at \url{https://github.com/mit-pdos/noria}.

\paragraph{Interface.}
Applications interface with Noria either through native Rust bindings, using
JSON over HTTP, or through a MySQL adapter~\cite{noria-mysql}.

\paragraph{Storage.}
The implementation maintains views in memory, and can maintain base tables
either in memory (the default) or on disk using using RocksDB~\cite{rocksdb}, a
key-value store based on log-structured merge trees.

\paragraph{Missing State.}
Noria does not store markers (``tombstones'') for missing results in a
materialization. Instead, it stores materialized results that are known to be
empty in hash tables alongside other (non-empty) materialized results. This
allows even empty results to be evicted to save space.

\paragraph{Upquery Bypass.}
When Noria encounters missing state and issues an upquery, it sends that
upquery directly to the root of the upquery path. This saves sending the
message along internal edges, but does not affect correctness as the
intermediate operators only forward the upquery upstream.

\paragraph{Batching.}
Noria uses several time-limited batching buffers to improve performance. Writes
to a base table are buffered for a few microseconds, and are emitted into the
dataflow as a single combined update to amortize lookup and processing costs at
operators like aggregations and joins. Noria also buffers upqueries in case
other misses for different keys along the same upquery path occur in quick
succession, and forwards them in a single batch.

\paragraph{Runtime.}
To multiplex I/O and compute, Noria uses Tokio~\cite{tokio}, a high-performance
asynchronous Rust runtime. Tokio manages a pool of threads that cooperatively
schedule thread domain processing (\S\ref{s:noria:partitioning}), query
handling, and control operations like adding and removing queries.

\paragraph{Fast Reads.}
Query handlers process clients' RPCs to read from external views. They must
access the view with low latency and high concurrency, even while a thread
domain applies updates to the view. To minimize synchronization, Noria uses
double-buffered hash tables for external views~\cite{evmap}: the thread domain
updates one table while read handlers read the other, and an atomic pointer
swap exposes new writes. This design can significantly improve read throughput
on multi-core servers over a single-buffered hash table with bucket-level locks
in workloads with high skew.
