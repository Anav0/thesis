Hello, and welcome!

This section is written for anyone who wants to understand \emph{roughly} what
is going on in my thesis, without necessarily understanding all the fiddly
technical bits. The running analogue I'll be using is one that I have, with
various degrees of success, used to explain my work to non-technical people over
the past five years. Hopefully, it'll be helpful to you as well!

Throughout the text, you'll find certain terms written in \textit{italics}.
Those are technical terms that are actually used in the thesis proper, and they
will arm you with some signposts to connect what you are reading here with the
thesis content.

\paragraph{The Library.}
%
Imagine a huge library that holds all the information for a website or app of
your choice. This could be Facebook, Twitter, Instagram, TikTok, Reddit, you
name it. It holds information about every user, every post, ever like, every
upvote, every comment, every picture, and every video. Every time you open said
website or app, some representative has to go to the library to collect all the
information relevant to whatever you are trying to view. If you are looking at
your Facebook timeline, the representative has to figured out who your friends
are, what they have posted recently, what comments there are on those posts,
etc. Similarly, if you are looking at a Reddit post, the representative must
gather the original post, but also any comments on that post, upvotes on those
comments, etc. The representative may also need to collect additional
information such as your name, and whether you have any new notifications or
messages. It's an exhausting affair. The library is a \textit{database}.

\paragraph{The Librarian.}
%
The representatives are not allowed to directly browse the library. Instead, the
library has a librarian who knows the library really well, and who answers
questions about the library's contents. When a representative wants to inquire
about something, they ask the librarian, who then scours the library to come up
with the answer to the representative's \textit{query}. The librarian is a
\textit{database engine}, which is often used interchangeably with ``database''.

\paragraph{Answering Questions.}
%
As you might imagine, some questions are easy to answer, whereas others may take
a very long time for the librarian to figure out the answer to. If someone asks
``what is Jon's email address?'', the librarian only has to look in the user
directory for the entry for Jon, and all the information is right there. That is, of
course, assuming that there is such a thing as a user directory which has
information ordered by the user's name (an \textit{index} whose \textit{key} is
the user's name). On the other hand, if someone asks ``how many people have
liked this post of Jon's?'', the librarian has a bigger task in front of them.
Even if there is a directory that lists likes by which post the like was for,
the librarian still has to count how many there are, which could (hopefully) be
a lot. Questions can even get so complicated that the librarian has to look
through every single like in the library to get the answer! For example, imagine
a representative asks ``what is the post with the most likes?''. To answer that
question, the librarian must know how many likes \textbf{every} post has, which
means they have to count the number of likes on every post. Ouch.

\paragraph{Writing Things Down.}
%
If we think back to the fact that these representatives are ultimately trying to
bring content to users who are sitting there waiting for the page to load, it
quickly becomes obvious that we need the librarian to answer questions
\textbf{very} quickly. Now, this librarian is very speedy indeed, but if the
questions get sufficiently complex, the answers still take time to find. So, one
day, the librarian has an idea. They realize that a lot of representatives are
asking the same small number of questions (the distribution of questions is
\textit{skewed}). For some reason, a lot of representatives want to know what
Robert and Frans are up to (very few bother checking what posts Jon has made
recently), and the librarian figures that if they can save themselves the
repeated trips, it'll save quite a bit of time. So, the librarian decides to
start \textbf{writing down} the answers they give out to representatives. When a
representative asks a question, the librarian first checks their list of
questions they've already found the answer to, and if the answer is there, they
don't have to leave their comfy chair! The librarian has decided to
\textit{materialize}, or \textit{cache}, query results.

\paragraph{Erasing Things.}
%
Unfortunately for the librarian, their plan has a flaw. Over time, their lists
grows so large that they're spending most of their time just reading through
their list to look for whether they've heard a particular question before! The
list is filled with questions no-one has asked in ages, which makes it hard to
find the questions that are asked a lot. Unfortunately, this librarian has a
very poor memory, so they don't actually remember which questions are asked
frequently, and which are not. So, the librarian decides to simply erase a bunch
of entries from their list at random. They figure that questions that are asked
a lot will be asked again soon anyway, and then end up on the list again,
whereas questions that aren't being asked much, well, are just going to stay off
the list. This is \textit{eviction}, and specifically \textit{randomized}
eviction. Other eviction strategies exist (like ``least recently used''), but
the work in this thesis uses randomized eviction.

\paragraph{Productive Humans.}
%
Unfortunately for the librarian, the library is every-changing. Every day,
representatives bring in piles and piles of new records that users have
produced. Likes, comments, photos, and more are coming by the boatload. Worse
yet, the representatives expect that those records immediately start being
included in the answers to all the questions that other representatives ask!
In the past, this wasn't too much of a problem\,---\,true, the librarian had to
file away the records that came in, which took some time, but at least by the
time they were looking for answers for the next representative, they would also
come across those new records and take them into account. But now that the
librarian is using their notebook, they're often not even looking at the
records, and so risk giving inaccurate answers! In other words, the answers in
the notebook grow \textit{stale}, and are no longer \textit{consistent} with the
data in the library.

\paragraph{Work-Work Balance.}
%
After thinking about their problem late at night when they have some downtime,
the librarian realizes that they need to be updating, or \textit{maintaining}
their notebook when they are filing away new records that arrive. They figure
that while this will make filing take longer, so many more representatives ask
questions than bring new records, that on balance the notebook should still
allow them to serve more representatives per day overall. The library
\textit{workload} is \textit{read-heavy}. The librarian still has to decide how
to split their time between servicing representatives that bring new records and
those that want to ask questions, but as long as both lines are shrinking, all
is good.

\paragraph{Throwing Away the Notebook.}
%
The first thing the librarian thinks of is to simply throw away
(\textit{invalidate}) the notebook any time a representative brings new files.
This works, but the librarian quickly discovers that this doesn't save much time
over not keeping a notebook at all. Since there is always a steady stream of new
files, the notebook barely gets a few entries in it before it has to be thrown
away! The librarian thinks there might be a way to only erase entries in the
notebook that are related to the newly filed records, but quickly eliminates
that option\,---\,the new records that come in frequently affect the questions
that most people have (new likes for Robert's newest post alongside questions
for how many likes Robert's newest post has). If the solution was to throw those
entries away, the librarian would still spend all their time counting how many
likes Robert's newest post has.

\paragraph{Updating the Notebook.}
%
While erasing a notebook entry one day, the librarian catches themselves
thinking that what they're doing makes little sense. A representative brought in
a single new record that was a like on Frans' latest post, and the librarian's
eraser currently hovers over an entry that says that the current number of likes
on that very same post is 999,999. The librarian erases the number, grins, and
before moving on to erasing the next entry puts down 1,000,000. This is genius!
Next time someone asks for the number of likes on that post, there's no need to
go count all those likes again\,---\,the entry in the notebook will still be
there \textbf{and} it will be correct. The librarian maintained the notebook
\textit{incrementally}, and thereby saved having to do a bunch of redundant work
later.

\paragraph{More. More! More!!}
%
Now that the librarian has discovered this little trick, they start looking for
other entries that can be updated in the same way. Unfortunately, while the
procedure is simple for some answers in the notebook, it is very tricky for
others. It's all well and good to add two numbers, but if Jon, who wasn't
following Robert before, starts following him, all of Robert's past posts now
have to be placed at the correct position in the list given as an answer to
``what posts have recently been made by people that Jon follows?''. The
librarian takes a break and ponders if there's a good way to solve this problem.

\paragraph{A Hierarchy of Notebooks.}
%
The next day, the librarian comes in with a plan, and pulls out a large sheet of
drawing paper. Each time a question comes in, the librarian maps out a
flow-chart for how they figured out the answer to that question. What
directories they looked things up in, in what order, and how the files from
those directories were combined. Then, the librarian keeps a separate notebook
for each step in that flow chart. Count the number of things in this directory
under this entry? Great, write that number down in one notebook. Look for all
the things in directory A that are also in directory B? Great, write down which
things were in both. All the way down to the final answer. If two questions
require similar steps, the librarian re-uses the overlapping parts of the flow
chart, and uses the same notebooks for the same steps. The librarian is mapping
out the \textit{dataflow} of the questions\,---\,how the entries in the notebook
relate to the data that's stored in the library.

\paragraph{Using the Flow-Chart.}
%
The librarian's insight only becomes apparent once the next batch of new records
are brought in by a representative. Now, instead of looking through all the
notebook entries, the librarian looks at where each new record would be filed.
Then, the librarian consults the giant flow-chart, and looks for the step in the
chart that indicates to look something up in that same place. If a new like
comes in, then the librarian looks in the flow-chart for a step that reads ``go
to the directory that holds likes''. Then, the librarian follows the edges of
the flowchart from that step. For each step, the librarian finds the entry in
that step's notebook that matches the new record, updates that entry to include
the new record, and then moves on to the next step. If a step is followed by
multiple parallel steps, the librarian does all of them, one after the other. By
the time there are no more parts of the flow-chart to follow, the librarian has
updated all the notebook entries that can possibly depend on the new record. And
crucially, without looking at anything unnecessarily! This is
\textit{incremental view maintenance using dataflow}, and is what \textit{Noria}
provides.

\paragraph{Common Knowledge.}
%
The librarian is now pretty happy\,---\,while it takes a little while to update
the notebooks to reflect new records that come in, it's not too bad, and a large
majority of all the questions that representatives come in with already have
up-to-date answers in the notebooks. Life is pretty good. But every now and
again, the librarian still has to answer questions whose answer does not appear
in a notebook. This is especially frustrating in cases where the librarian is
pretty sure that they found the answer some time in the past, but has since
erased the relevant entry to save space in the notebook. It feels like there
should be a way to cobble the answer together from related tidbits in other
notebooks that may still hold parts of the answer, rather than having to go all
the way back to the library shelves and do all that tedious manual counting. If
Robert's post's like count is still in \textbf{some} notebook, then the
librarian shouldn't have to count the likes again just because there isn't an
entry for the specific question the representative asked.

\paragraph{Suspiciously Similar Flow-Charts.}
%
After going through the steps of answering some of these questions where it
feels like at least part of the answer lies in a notebook somewhere, the
librarian starts to notice a pattern. When drawing out the flow-chart steps for
the new question, there's nearly always overlap with steps from some other
questions. And while writing into the notebooks for those shared steps, there's
almost always an entry with the exact same answer already present. The librarian
realizes eventually that this is not actually so surprising\,---\,if two
questions both ultimately require that the librarian count the number of likes
for one of Robert's post, then they will both share the steps related to that in
the flow-chart. And the result will in both cases end up in the same entry in
the relevant notebook\,---\,the one for that post.

\paragraph{Flow-Charts in Reverse.}
%
Following this observation, the librarian decides to try something. The next
time a question comes in for which the notebooks do not have an answer, the
librarian maps out the flow-chart for answering that question as usual. But
instead of then following the flow-chart from the top (``start by opening the
likes directory''), the librarian follows the flow-chart \textbf{in reverse}.
The librarian first looks in the answer notebook at the end of the question's
flow-chart, and if the answer isn't written down there, then goes up to the
notebook for the second-to-last step of the flow-chart. If the relevant entry is
written down there, then the librarian can just take what's written there, do
the last flow-chart step, and then have the answer for the representative's new
question! If that notebook also has no relevant information, the librarian
continues ``up'' the flow-chart until they either find an entry, or the
flow-chart says to look in one of the data directories, where the relevant data
is guaranteed to reside. What the librarian is doing is an
\textit{upquery}\,---\,a reverse lookup in the dataflow for information that
isn't \textbf{quite} refined enough to answer the question, but is better than
having to consult the entire data directory. Upqueries are particularly
attractive because they allow the librarian to keep only a single flow-chart,
rather than keep multiple ``what to do if this notebook doesn't have the
information?'' flow-charts.

\paragraph{Conclusion.}
%
This is as far as this analogue will go. It's not perfect, but it should give
you sufficient working knowledge of the problem area that this thesis tackles.
In particular, the contributions of this thesis is the notion of ``upqueries'',
as exemplified by the last paragraph. All the techniques from the preceding
paragraphs already exist in past related work.

You may wonder about the lack of the word ``upquery'' in the thesis title, and
what ``partial state'' means. This is one place where the library analogue
starts to break down. An approximate explanation is that partial state is what
enables the librarian to erase entries from notebooks, and upqueries are an
important part of how to make having such erased entries practical. And while
the ability to erase things may seem trivial, it turns out that the librarian's
flow-chart approach gets tricky when information may be missing from notebooks
part-way through. Especially since most databases (``libraries'') have multiple
concurrently executing ``librarians'', which all share notebooks and need to
ensure that they do not step on each other's toes or overwrite each other's
work!

I hope that was helpful. Thank you for reading!
