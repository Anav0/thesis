Noria implements the partially stateful dataflow model for incremental view
maintenance in databases. It focuses on building a better database backend for
read-heavy applications where a long-running dataflow program maintains any
number of materialized user-defined views, specified in SQL. Noria uses joint
query optimization techniques to find ways to integrate new views and queries
with the running dataflow. The system is also highly concurrent and distributed,
and supports sharding cliques of operators to share resource costs and increase
sustainable throughput.

Dataflow is a broad term, so I want to take a moment to discuss Noria's specific
dataflow implementation. Noria takes SQL queries from the application, and folds
them into a single, unified dataflow program. The dataflow is a directed,
acyclic graph, with the base tables at the ``top'', and application-accessible
views at the ``bottom''. The nodes in between represent the SQL operators that
make up the query of every view. Reads (\texttt{SELECT} queries) access
materializations at the leaves only, while writes (\texttt{INSERT},
\texttt{UPDATE}, and \texttt{DELETE} queries) flow into the graph at the roots.

After a write has been persisted by the base table operator, it flows into the
dataflow graph as an ``update'', following the graph's edges. Every operator the
update passes through processes the update according to the operator semantics,
and emits a derived update. Eventually the update arrives at a leaf view and
changes the state visible to reads through the leaf's materialization. Updates
are \emph{signed} (i.e., they can be ``negative'' or ``positive''), with
negative updates reflecting revocations of past results, and modifications
represented as a negative-positive update pair.
