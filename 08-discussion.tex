This thesis presents the partially stateful model, as well as its implementation
in Noria. And while the model is complete in isolation, there are a number of
secondary considerations, features, and alternatives that are worth discussing.
Those are discussed in this chapter.

\section{When is Noria not the Answer?}

Noria aims to improve the efficiency of certain classes of database-backed
applications, but is not a one-size-fits-all solution. Noria's materialized
views generally, and partial state specifically, are built specifically for
applications that:

\begin{enumerate}
  \item Are \textbf{read-heavy}. Noria's design is centered around making reads
    cheap, often at the cost of write performance. For workloads where writes
    are as frequent, or more frequent, than reads, other systems will work
    better.
  \item Tolerate \textbf{eventual consistency}, at least for large parts of the
    application's workload. Much of Noria's performance advantages over other
    systems stems from the relaxed consistency model. If much of the
    application's workload requires stronger consistency guarantees, there is
    little for Noria to speed up.
  \item Experience \textbf{good locality}. If the application's access patterns
    are all completely uniform, caching is much less impactful unless \emph{all}
    results are cached. In that case, partial state, and the complexity it
    introduces, provides little value. Instead, Noria works best if data and
    access distributions are skewed, and demonstrate good temporal and spatial
    locality.
  \item Have \textbf{non-trivial computed state}, both in size and complexity.
    If all computed state fits in a small amount of memory, a materialized view
    system without partial state would work just as well. If all queries are
    simple point queries without aggregations or joins, Noria's incremental
    cache update logic is unnecessary, and a simpler cache invalidation scheme
    may work better.
\end{enumerate}

It is also worth noting that Noria may not perform as well as a fully developed,
manually tuned backend caching system. While Noria would allow the removal of
caching logic from the application, its general-purpose architecture may miss
out on application-specific optimizations implemented by a tailor-built system.

\section{Emulating Partial State}
\label{s:disc:emulating}

A natural question to ask is whether the benefits of partial state can be
achieved without the complexity that upqueries introduce. In particular, can a
dataflow system that supports only full materialization emulate partial state
effectively? Thoroughly exploring the answers to this question may be worth a
thesis in its own right, but some of the more obvious approaches are discussed
below.

\subsection{Lateral Joins}

The commercial materialized view stream processor Materialize~\cite{materialize}
supports \emph{lateral} joins~\cite{lateral-join}, which is described as

\begin{quote}
  [A] join modifier [that] allows relations used in a join to ``see'' the
  bindings in relations earlier in the join.
\end{quote}

In particular, lateral joins let the application author write a query that has
access to the contents of some unrelated table. For example,
Listing~\vref{l:emulate-partial-vote} shows how a lateral join can be used to
emulate a partially materialize vote count view like the one from
Listing~\vref{l:votes}. The idea is to have a table of ``filled'' keys, and have
the results only for those keys be included in the final materialized view.

\begin{listing}[h]
  \begin{minted}{sql}
CREATE MATERIALIZED VIEW VoteCount AS
SELECT article_id, votes FROM
  (SELECT DISTINCT article_id FROM queries) filled,
  LATERAL (
      SELECT COUNT(*)
      FROM votes
      WHERE article_id = filled.article_id
  );
  \end{minted}
  \caption{Using a Materialize lateral join to emulate partial state in vote.}
  \label{l:emulate-partial-vote}
\end{listing}

This approach works well to emulate partial state in simple situations, but
requires significant manual effort for a large application. In Lobsters, for
example, the application author must re-write their applications to use such
lateral joins, and must include application logic to maintain the auxiliary
tables used to indicate what keys are materialized. It may be possible to
automate this process, though doing so requires research.

Effort notwithstanding, emulating partial state in this way also presents an
``all or nothing'' choice for applications for a given key. Either, all state
for that key is computed, or none of it is. With partial state, the state for a
key in the ultimate materialized view can be evicted without also evicting the
current vote count. The former may be significantly larger than the latter,
since it includes other columns, but is cheap to recompute. The latter on the
other hand is small, but potentially expensive to re-compute.

\subsection{State Sharing}

Partial state allows a single query of the form \texttt{WHERE x = ?} to satisfy
lookups for any value of \texttt{?}. Without partial state, the system has two
options: to remove the filter on \texttt{x} from the query and filter after the
fact, or to instantiate a separate query for each concrete value of \texttt{?}
supplied by the application. The former uses a significant amount of memory, but
is also complicated to get right; \texttt{x} may for example affect what values
are aggregated together. The latter is simpler, and uses less memory, but
requires duplicating the dataflow operators for each query, and keeping separate
state for each one.

Recent work introduced arrangements~\cite{arrangements} as a way to mitigate
this problem. Arrangements allow sharing indexes and state across related
operators to avoid duplication. However, even with arrangements, the system may
execute the same computation over a given input record more than once if it is
needed by more than one instance of a query. Furthermore, eviction is made more
difficult with arrangements, as the system must update the entire arrangement to
accommodate any new or evicted parameter value. Noria supports joint query
optimization~\cite{noria}, which combined with arrangements could reduce much of
the duplicated effort by instantiating each query multiple times, though this
does not improve the eviction process.

\section{Consistency}

Noria provides weaker consistency guarantees than many existing dataflow and
view materialization systems. This has implications for how applications use
Noria, and what behavior the application may observe.

\subsection{Write Latency as Staleness}

By design, Noria's read and write paths are disconnected from one another: reads
can usually proceed even if the write path is busy. This is both the reason why
Noria's read performance is so high, and why it gives weaker consistency
guarantees that competing systems. For example, on a 32-core machine, the
application may experience a write throughput ceiling at a few hundred thousand
updates per second, as the write path is processed by only a small number of
cores. Meanwhile, reads can happen across any number of cores; even if the write
path is entirely saturated, Noria may be able to handle millions of additional
reads per second.

While a saturated write path does not slow down the execution of queries whose
results are materialized, it does affect the read path in two important ways:
miss-to-hit time and result staleness. If a query misses, the dataflow must
compute and populate the missing state so that the read can proceed. This is the
same dataflow that handles writes, so the time until the missing read hits
instead will increase if the dataflow is busy. Similarly, while queries that do
not miss can proceed immediately, the returned results will not reflect updates
that have not yet been processed by the dataflow. Therefore, if the dataflow is
busy, the time between when an update is issued and when it is reflected in
later queries will increase.

\subsection{Transactions}

Web applications sometimes rely on database transactions, e.g., to atomically
update precomputed values. Noria does not implement transactions, though its
support for derived views often obviates the need for them. For example, web
applications often use transactions to keep denormalized schemas synchronized: a
``like count'' column in the table that stores posts or an ``average rating''
column in the table that stores products. Noria obviates the need for such
denormalization, and the transactions needed to maintain them, by automatically
ensuring that computed derived values are kept up to date with respect to the
base data.

\subsection{Stronger Consistency}

Noria is eventually consistent, and so is the partial state implementation
outlined in this thesis. That said, adding partial state to a system with
stronger consistency guarantees should not require extensive changes. In fact,
parts of the design could likely be simplified; union buffering, for example,
would likely no longer be necessary, and could be replaced with some kind of
multi-versioned concurrency control.
