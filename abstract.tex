\documentclass[fontsize=12pt,paper=letter]{scrartcl}

% for \sfrac
\usepackage{xfrac}

\begin{document}
\begin{titlepage}
  \begin{center}
    \textbf{\large Partial State in Dataflow-Based Materialized Views}\\
    \vspace{0.5\baselineskip}
    by\\
    \vspace{0.5\baselineskip}
    {\large Jon Gjengset}\\
    \vspace{0.5\baselineskip}
    \begin{addmargin}[0.15in]{0.15in}
      \centering
    Submitted to the Department of
    Electrical Engineering and Computer Science
    on \today{}
    in Partial Fulfillment of the Requirements for the Degree of
    Doctor of Philosophy in Computer Science
    \end{addmargin}
  \end{center}

  \begin{flushleft}
  ABSTRACT
  \vspace{0.5\baselineskip}

  This thesis proposes a practical database system that lowers latency and
    increases supported load for read-heavy applications by using
    incrementally-maintained materialized views to cache query results. As
    opposed to state-of-the-art materialized view systems, the presented system
    builds the cache on demand, keeps it updated, and evicts cache entries in
    response to a shifting workload.

  \vspace{0.5\baselineskip}

  The enabling technique the thesis introduces is \textit{partially stateful
    materialization}, which allows entries in materialized views to be
    \textit{missing}. The thesis proposes \textit{upqueries} as a mechanism to
    fill such missing state on demand using dataflow, and implements them in the
    materialized view system Noria. The thesis details issues that arise when
    dataflow updates and upqueries race with one another, and introduces
    mechanisms that uphold eventual consistency in the face of such races.

  \vspace{0.5\baselineskip}

  Partial materialization saves application developers from having to implement
    ad hoc caching mechanisms to speed up their database accesses. Instead, the
    database has transparent caching built in. Experimental results suggest that
    the presented system increases supported application load by up to
    $20\times$ over MySQL and performs similarly to an optimized key-value store
    cache. Partial state also reduces memory use by up to $\sfrac{2}{3}$
    compared to traditional materialized views.

  \vspace{\baselineskip}
  Thesis Supervisor: Robert Tappan Morris\\
  Thesis Co-Supervisor: M. Frans Kaashoek\\
  Thesis Readers: Sam Madden and Malte Schwarzkopf
  \end{flushleft}
\end{titlepage}
\end{document}
