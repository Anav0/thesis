Where do I even start to write an acknowledgments section for the past six
years, much less the many many early years that got me to MIT and PDOS in the
first place? Do I go chronologically? By how much of a difference they've made
to this thesis in particular? Or to my life more generally? I'm genuinely more
intimidated by this section than by the rest of this thesis.

I think where it feels right to start is to thank my parents for putting up with
my never-ending exploration of everything computer related. They may have rarely
understood much of what I was doing, but they encouraged me to continue
``geeking out'', and boy did I. And the appendix that explains Noria in simpler
terms would never have existed were it not for my mom's endless desire to
understand what I was working on combined with her lack of interest in listening
to long-winded technical descriptions.

As far as this thesis is concerned, I would never have gotten here without the
help of Malte Schwarzkopf, who worked tirelessly by my side on Noria for many
years. Without him, the Noria paper would have never seen the light of day, and
many of Noria's key features, like SQL support and joint query optimization,
would not exist. Indeed, the name Noria was his idea!

The same can be said for my advisors, Robert Morris and Frans Kaashoek. Since I
first joined MIT, they have been patient, helpful, understanding, and insightful
at every turn. No matter the topic or severity, their door has always been open.
And despite my frequent detours into side-projects, or my never-ending desire to
TA ``just one more class'', they never forced my hand, and instead let me find
my own way back to research productivity. I knew from when I first met Robert
during MIT visit days that I wanted to work with him\,---\,he has this uncanny
ability to digest large amounts of complex technical information and come up
with succinct, innocent-sounding questions that cut right to the core of some
previously unidentified technical challenge, design flaw, or incorrect
assumption. This, combined with his quiet, dry-wit demeanor makes Robert the
person with the highest quality-per-word ratio I have ever met. Meanwhile, Frans
excels at holistic thinking\,---\,he has a deep understanding of which
narratives work, and which do not, without losing sight of the technical
contributions, and this thesis wouldn't convince anyone of anything without
Frans' guidance. He's also a social being who knows \emph{everyone}, and is a
delight to be with in social settings. I couldn't have wished for a better team
to guide me than the two of them together.

When it came time to put together my thesis committee for this work, I was
delighted that I got Sam Madden on board. While I have not interacted much with
Sam up through the years, he seems to always be in a good mood, and happy to
chat. I was excited to have his experienced database eyes on this work both to
make sure I did not accidentally re-invent a subpar version of a system that the
database community invented in the 1980s, and to critically evaluate whether
Noria might actually work alongside ``real databases''. Luckily, if the defense
is anything to go by, he seems to think this is a pretty good idea!

The Noria project would not be what it is today without Eddie Kohler from
Harvard. He was key to getting the many Noria papers up through the years over
the finish line, provided invaluable experience from his past database work, and
inspired a number of interesting research directions for Noria. Eddie also
instilled in me the importance of thinking in terms of system invariants, and
even initially suggested the term ``upquery''!

There have also been countless students involved with the Noria project over its
lifetime, including Jonathan Behrens, who prototyped transaction support in
Noria; Martin Ek, who wrote Noria's durable storage backend; Lara Timb\'{o}
Ara\'{u}jo, Alana Marzoev, Samyukta Yagati, and Jackie Bredenberg who extended
Noria to support privacy and security policies; Nikhil Benesch who was part of
early Noria design discussions; Gina Yuang, who designed a fault tolerance
scheme for Noria; and Jonathan Guillotte-Blouin, who implemented preliminary
support for range queries. Their efforts, and that of others, not only made
Noria better in a technical sense, but also added to the enthusiasm around the
project, and inspired my continued work on it. Without their involvement, the
project may have died long ago.

I'd also like to acknowledge the influence of Frank McSherry on this work. Not
only was his take on the dataflow model a big factor in how Noria was designed,
but being able to bounce ideas off of him has been invaluable in getting a
sobering outside perspective. True to his name, Frank's advice is always direct
and honest, and this work is better as a result.

The Rust programming language community in general, and the Tokio project
maintainers in particular, have also contributed enormous technical value to
Noria. Not only by the tools and ecosystem they have built, but also through a
number of technical discussions that have improve the implementation of Noria
far beyond what I could have done on my own.

Beyond Noria specifically, I cannot overstate the impact my research group,
PDOS, had on my experience at MIT. Akshay, Amy, Anish, Atalay, Cody, David,
Derek, Frank, Inho, Josh, Lily, Neha, Tej, Zain, I will forever be grateful for
the interesting discussions I've had with you all over the years. And especially
my roommate Tej Chajed, with whom I've spent countless hours debating every
topic on earth from system designs to the point of research to the meaning of
words. Outside of PDOS, I am also indebted to Leilani Battle and Max Wolf for
opening up my eyes to the joy of board games, which have sustained me through
many a gray winter night, and to my other roommate, Davis Blalock, whose
never-ending curiosity was always a source of delight and fun conversation.

I got into research in the first place in large part due to Phil Stocks and
Warren Toomey from Bond University in Australia, and Kyle Jamieson, Mark
Handley, and Brad Karp from University College London, all of whom inspired me
with their wisdom, insight, curiosity, and technical skill. Each one nudged my
appetite for research, and encouraged me down the path I ultimately took. I also
want to give a nod to Martin Kirkengen, my middle school math teacher, who
showed me the joy of figuring out how things work behind the scenes, and Dave
Stanley, my middle school English teacher, who taught me the value of not being
a smart-ass all the time.

Part of what let me keep my sanity through my PhD was my continued work at
Oksnøen, an activity summer camp for kids in Norway. For several weeks each
year, I'd unplug my computer and go there to take my mind off \emph{everything}.
Being outside, active, and social there recharged my mental batteries in a way I
cannot stress the importance of enough. And I owe my ability to complete this
work to my friends among the counselors there, and the children who come back
year after year and put joy in my heart.

These last few years, I also owe a great deal to my girlfriend, Talia Rossi, who
has always been there to listen when things have gotten tough, and to remind me
to not get too absorbed when things got hectic. Not to mention her understanding
and compassion when I sequester myself by my desk in an attempt to finish this
thesis.

And finally, I'd like to thank you who's reading this. Over the years it's taken
me to write this PhD, one of my primary drivers has been to build something that
would be useful to other people. If you're reading this, it means that I can
hopefully contribute something interesting to your life, and that makes it all
worth it.
