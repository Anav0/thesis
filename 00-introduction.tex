\section{Motivation}

\begin{enumerate}
 \item Why is materialization useful for real-world use-cases?
 \item In what ways are existing materialization techniques lacking?
 \item How does partial state improve the usefulness of materialization?
\end{enumerate}

Caching.
OSDI paper.
Caching is labor intensive + error prone (Facebook memcached paper).
DB view materialization (like DBToaster) ``all or nothing''.
Commercial DB view materialization limited and slow (ref OSDI paper).
By folding the cache into the database, database knows how to keep it up
to date\,---\,avoids having that logic threaded through the application.
Partial state (and caching in general) enables \textbf{selective} materialization.
Materialization is generally useful when reads are more common than
writes, since it shifts the computation from reads to writes.
But, without partial state, materialization is an ``off/on'' option —
either you materialize all the results for a query, or none [TODO: not
quite true — more subtlety here — depends what you mean by ``query''
(e.g., parameterized or not)].

The dataflow architecture has seen a resurgence in recent years, with systems
like Kafka\cite{kafka}, Spark\cite{spark,spark-streaming}, Flink\cite{flink}, and
Naiad\cite{naiad} all seeing interest from academia and industry alike. The term
``dataflow'' covers a wide range of systems: streaming systems (like Kafka),
stream processing systems (like Spark and Flink), and dataflow computation
systems (like Naiad). These systems vary in their design goals, their intended
uses, and their system architecture. Yet they all share the property that they
take data as \emph{input}, and feed processed data forward through a graph of
\emph{operators}.

This data-flow design is attractive for ready-heavy applications where the
computation is relatively fixed over time, and it is the data that changes.
We can store the results of the computation to serve reads quickly, and use the
known data dependencies to keep those results efficiently up to date as the data
changes. Explicitly modeling the compute and data flow this way also allows
dataflow systems to easily scale to multiple cores or physical hosts; any
dataflow edge can be realized as a function call, a thread synchronization
point, or a network channel, and the model still ``works''.

These systems all face a decision when it comes to \emph{stateful} operators\,
---\,operators that must maintain some state in order to process their inputs.
To compute the result of a join for example, the values in the ``other side'' of
the join must be available to the operator that performs the join.
Traditionally, dataflow systems have made one of two decisions: ignore such
operators, make all the state the operator needs available, or keep only a
subset of the state. Each of these caters to a particular type of application.
Non-stateful dataflow is primarily useful as a messaging fabric. Fully stateful
dataflow works well for applications that operate over small data sets, or where
the application's working set spans nearly the entire data set.

A third intermediate solution called \emph{windowing} is an option for some
applications. Windowed state only keeps \emph{recent} data, which ensures that
the state remains small over time. However, this also means that old state is
not reflected in computational outputs. If an application needs to compute over
the full application state, such as all users or all articles, windowing does
not work. Windowing is primarily used for analytics applications, where
historical data is less relevant.

But unfortunately, not all applications fit into one of these categories. In
particular, user-facing applications whose working set is significantly smaller
than their total dataset size are not well served by these options. Stateless
operation is not feasible, since evaluating the dataflow from scratch each time
would incur significant extra latency. Fully stateful operation is similarly
unattractive\,---\,computational resources would need to be provisioned for
computing over all the application's data, even though only a small subset of
the computation output is observed. And windowing is frequently not applicable
to these applications; since users may request data that lives outside the
window, that data must still be available.

This thesis presents the Noria dataflow system; a dataflow system that supports
\emph{partially-stateful dataflow}. In Noria, operators act as though they have
access to the full state of their inputs, while in reality that state is lazily
constructed behind the scenes; a given piece of the input state for an operator
is only produced and stored when the operator asks for that piece. From that
point forward, the state is continuously updated to reflect new data added to
the system. If an operator only accesses part of its input state, the remaining
parts are not computed or stored.

This approach provides a number of benefits. First, its memory use is
\textbf{proportional to the working set} of the application, rather than to the
size of the data. Second, it works for applications that \textbf{cannot use
windowing}. Third, it allows the system to \textbf{eagerly discard}, and avoid
computation for, data that later operators have never needed, as long as that
data can later be re-produced. And finally, it allows the application to
\textbf{selectively evict} from stateful operators as the working set changes.

Another key advantage of partial state is that it makes it possible to extend a
running dataflow program \textbf{lazily}. Noria can cheaply accommodate new
segments of dataflow by instantiating the new dataflow as initially empty. That
new dataflow is then populated through application activity, rather than by
incurring a large upfront cost.

My thesis will cover the design and implementation of partially stateful
dataflow in Noria in detail, including several key components that were only
briefly sketched or not present at all in the earlier OSDI paper on
Noria\cite{noria} that I co-authored. I will discuss the specific problems that
arise in depth, and provide solutions to those problems.

\section{Challenges}
\section{Approach}
\section{Noria}
\section{Contributions}

The contributions of my thesis, subject to this proposal, will be:

\begin{itemize}
 \item An algorithm for implementing upqueries.
 \item Support for partial in sharded, complex applications.
 \item Key correctness invariants for partially stateful dataflow.
 \item Case analysis of the issues that arise when introducing partial state to
	 a distributed, high-performance stateful dataflow processing system.
 \item Techniques for overcoming those issues while preserving system
	 correctness, performance, and scalability.
 \item Micro and macro evaluations of the performance and memory impact of
	 introducing partial state to an application's dataflow.
\end{itemize}

\section{Dissertation Outline}
