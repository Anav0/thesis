This chapter provides an overview of existing systems and their relation to
Noria and partial state.

\section{Materialized Views}

Database materialized views~\cite{materialized-views} were originally devised to
store expensive analytical query results for quick recollection. Unfortunately,
commercial databases' materialized view support is limited, and views must
usually be rebuilt from scratch when the underlying data
changes~\cite{materialized-view-selection-sql-server,
mssql-materialized-view-restrictions-blog,
mssql-materialized-view-restrictions}.

The key to usable materialized views is how they are maintained as the
underlying data changes. Rather than throw away the current materialized query
results, a good materialized view system should only perform the work needed to
\textit{incrementally} update the materialized results. This has been the
subject of considerable research in the past few decades.
\citeauthor{materialized-survey} gives a good survey of the current
landscape~\cite{materialized-survey}.

Modern incremental view maintenance (IVM) techniques tend to be based on
\textit{delta queries}. Delta queries are algebraically derived queries that
give efficient relational expressions for computing changes to a (materialized)
view given a set of changes to the underlying data. The current state-of-the-art
is \textit{Higher-Order IVM}, in which the system derives multiple, recursive
such delta queries for each view, and materializes and maintains intermediate
delta query results as well~\cite{dbtoaster, hotdog}. Recent work proposes
techniques for mitigating the memory overhead of such intermediate
materializations by instead materializing smaller auxiliary state from which the
necessary values can then be efficiently produced when
needed~\cite{memory-efficient}. Sadly, few of these solutions have been adopted
in commercially available databases. Unlike Noria, these systems focus on
long-term maintenance of analytics queries\,---\,they do not provide mechanisms
for fast reads and do not support eviction.

Noria's dataflow resembles Higher-Order IVM, including the materialization of
intermediate results. Noria's algorithm to determine what dataflow to use to
compute changes to each view is naive compared to delta queries, and could
likely benefit from the techniques in the aforementioned work.

Pequod~\cite{pequod} and DBProxy~\cite{dbproxy} provide materialized views that
also support partial materialization in response to client demand. However,
Pequod is limited to static queries specified in a datalog-like language, and
DBProxy does not support incremental view maintenance. And neither system shares
state nor processing across views.

\section{Caching}

Application-level caching is often implemented in an ad hoc fashion, and is the
source of many application errors~\cite{ad-hoc-caching}. In particular, such ad
hoc system often fail to invalidate or update the cache as the underlying data
changes, leading to permanently stale entries. Researchers and industry teams
alike have therefore attempted to build systems to automate cache maintenance.

Authors of large applications often build their own custom caching
infrastructure that solves their immediate needs~\cite{facebook-memcache,
flannel}, but does not provide a ready-to-use solution for other developers who
face similar issues. These custom-built solutions tend to implement only the
minimum functionality the authors need at the time, and forego more complicated,
but nonetheless useful features like incremental cache updates. Noria presents a
``plug and play'' solution specifically for query result caching for many
applications.

The research community has also produced several systems that aim to provide
more general-purpose transparent caching. TAO~\cite{tao} and
TxCache~\cite{txcache} implement automated query result caching, but do not
support incremental in-place cache updates like Noria.
CacheGenie~\cite{cachegenie} implements a trigger-based middleware cache for
object-relational mapping frameworks, and supports in-place cache updates, but
is limited to only specific operations supported by the framework. In contrast,
Noria transparently speeds up regular SQL queries, and does not require the
application to use a particular database abstraction framework.

In the database literature, database caching front ends are sometimes referred
to as \textit{Cache-Augmented SQL} systems. And there, like with all caching
systems, the primary concern is consistency\,---\,\emph{some} mechanism must
ensure that the cache remains up to date as the underlying data changes.
Research in this space tends to focus on augmenting the key-value systems that
stores cache entries so that the application can correctly manage races between
database updates and cache invalidations~\cite{facebook-memcache,
casql-consistency, casql-consistency-thesis}. Noria instead integrates the cache
management into the database, which allows the cache entries to be incrementally
updated, automatically, in-place, albeit with eventual consistency.

A related approach is \textit{mid-tier database caching}, in which subsets of
the database are replicated onto the hosts that run the application's code. This
allows certain queries to be run locally without interacting with the remote
database backend~\cite{mtcache}. While the approach is appealing in that some
database queries can avoid traversing the network, it does not provide the same
speedups that query result caching provides.

\section{Dataflow}

A wide range of dataflow and stream-processing systems exist that excel at
data-parallel computing~\cite{dryad, naiad, storm, heron, flink, millwheel,
spark-streaming, stanford-stream, s-store, cloud-dataflow}. However, these
systems cannot easily serve web applications directly. They only achieve
low-latency incremental updates at the expense of limiting how much state they
keep by \textit{windowing}, which results in incomplete results, or by keeping
full state in memory. Partial state allows Noria to lift this restriction.
Furthermore, these systems generally provide no mechanism for accessing computed
state except through the dataflow or by integrating with additional external
systems, which adds latency.

Many existing system are also limited to a fixed set of queries defined when the
system starts, and cannot easily adopt query changes. Some dataflow systems do
support Noria-like dynamic changes to the running dataflow~\cite{ciel, ray}, but
without support for demand-driven partial state these systems must either fully
compute results when the dataflow is extended, or have new dataflow only take
into account subsequent updates.

Some developers use, or consider using, a streaming fabric like Apache
Kafka~\cite{kafka} to build their own view maintenance pipeline~\cite{nyt-kafka,
samza-blogpost}. However, at the time of writing, no general-purpose system
exists based on such a pipeline that achieves the performance and flexibility of
Noria.

Differential dataflow~\cite{differential-dataflow}, and its instantiation in the
commercial product Materialize~\cite{materialize}, bears a striking resemblance
to Noria at first glance. In particular, it uses dataflow to produce
automatically-maintained materialized views over SQL queries. However,
Materialize does not implement partial state, and must therefore maintain
similar queries independently (which misses out on opportunities for shared
compute and state) or fully materialize query results (which uses more memory).
The authors behind Materialize have proposed partial solutions to some of these
challenges, which are discussed in \S\ref{s:disc:emulating}.
